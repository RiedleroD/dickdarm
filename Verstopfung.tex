\documentclass[11pt]{article}
\title{\textbf{Verstopfung}}
\author{Riedler}
\date{}

\usepackage{geometry}
\usepackage{hyperref}
\geometry{
 a4paper,
 total={170mm,257mm},
 left=25mm,
 right=25mm,
 top=25mm,
 bottom=25mm
}

\hypersetup{colorlinks=true,urlcolor=blue}

\begin{document}

\maketitle
\thispagestyle{empty}

\section*{Himmelsschmerz}

Ist man doch oft verdrossen,\\
sieht man zum Himmel auf.\\
Dann wird kurzwegs beschlossen:\\
Dort oben wär ichs auch.

\section*{Fremd im Heimatsland}

Zieht man in unbekannte Reiche\\
dann ist man fremd, und nicht der Gleiche.\\
Und zieht man dann zum Heim zurück,\\
dann findet man dort auch kein Glück.\\
Denn war man mal im Ausland drin,\\
ist man Ausländer ohnehin.

\section*{Web 3 with sticks and stones\textsuperscript{[\href{https://mas.to/@Riedler/109154135549306406}{1}]}}
"Sticks and stones may break my phones,\\
but they'll never hurt web 3.",\\
he said while bound by giant loans,\\
cursed to nevermore be free.

\section*{Pommes-Schratze}

Ich möchte Pommes von dem Stand,\\
dort am rechten Straßenrand.\\
Mit Ketchup hab ichs liebend gern,\\
auch Mayo läge mir nicht fern.\\
Doch setz dich nicht so nahe her,\\
beim Essen geht es wild umher.\\
Da kann es sein, dass ich mal patze.\\
Bin ja auch der Pommes-Schratze.\\
\\
anm.: Kontext ist kaum vorhanden, aber \href{https://mas.to/@Riedler/108709490168652329}{hier} ist der Link.

\end{document}

